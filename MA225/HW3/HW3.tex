
\documentclass[11pt]{letter}
\usepackage{amsmath,amsthm}
\usepackage{amsfonts}
\usepackage{amssymb}
\usepackage{enumerate}
\usepackage[margin=.9in]{geometry}
\usepackage{graphicx}
\newtheorem*{theorem*}{Theorem}
\newtheorem*{IBP}{Integration by Parts}
\newtheorem{claim}{Claim}
\theoremstyle{definition}
\newtheorem{defn}{Definition}
\newtheorem{PBI}{Proof by Induction}

\begin{document}
\pagestyle{empty}

{\Large MA 225 Problem Set 3: induction 1}\\

\begin{description}
	\item[facts and definitions] You will need the following definitions and facts (at some point):
			\begin{defn}
				We'll call strings of the symbols $a$ and $b$ {\em words}. Consider a class of words, the {\em legal words}, defined as follows. $aba$ is a legal word. If $W$ is a legal word, then so are $abW$, $aWb$, $Wab$, $baW$, $bWa$, and $Wba$. No words other than those obtained in this way are legal.
			\end{defn}
			\begin{defn}
				The {\em binomial coefficients} are a collection of natural numbers $B_{n,k}$, defined for a pair of nonnegative integers $n,k$ with  $0\leq k\leq n$, as follows:
				$$B_{n,k}=\begin{cases}1 & \text{ if } k=0\text{ or }k=n\\B_{n-1,k-1}+B_{n-1,k} & \text{ if }1\leq k\leq n-1\end{cases}$$
			\end{defn}
			
			\begin{defn}
				{\em Higher derivatives} are defined as follows: the zeroth derivative of a function $f$ is $f$ itself; we write
					$$\frac{d^0}{dx^0}f(x)=f(x)=f^{(0)}(x)$$
				For $n\in \mathbb{N}$, we define the $n^{th}$ derivative as the derivative of the $(n-1)^{st}$ derivative:
					$$\frac{d^n}{dx^n}f(x)=\frac{d}{dx}\left[\frac{d^{n-1}}{dx^{n-1}} f(x)\right]$$
					$$f^{(n)}(x)=\left(f^{(n-1)}\right)'(x)$$
			\end{defn}
			
			\begin{IBP}
				For any two differentiable functions $f$, $g$, we have
				 $$\int f(x)g(x)\ dx = f(x)\left[\int g(x)\ dx\right]-\int f'(x)\left[\int g(x)\ dx\right]\ dx,$$ provided we adopt the convention that the constants of integration in both instances of $\int g(x)\ dx$ must be the same.
			\end{IBP}

	\item[exercises]  These problems don't require you to write proofs.
		\begin{enumerate}
			\item Compute $B_{n,k}$ for $0\leq n\leq 6$. \\
                        \begin{tabular}{||c|c|c|c|c|c|c|c||}
                          \hline
                        ${}$&k=0&k=1&k=2&k=3&k=4&k=5&k=6 \\
                          n=0 &1 \\
                          n=1&1&1 \\
                          n=2&1&2&1\\
                          n=3&1&3&3&1\\
                          n=4&1&4&6&4&1\\
                          n=5&1&5&10&5&1\\
                          n=6&1&6&15&20&15&6&1\\

                            
                          \hline
                          \end{tabular}  
                          
			\item Explain why the definition given for $B_{n,k}$ actually constitutes a definition; that is, why we can compute $B_{n,k}$ for any choice of $n,k$ with $0\leq k\leq n$. We can do this because one can work their way up from $B_{0,k}$, which is 1 and continue to get the rest of their solutions.
			\item Identify, explain, and correct any correctable flaws in the following proofs:
                          The problem with this proof is we don't see $n^2+n$ is odd for n, instead it tells us that this is true, when in fact this isn't as 1+1=2 which is even. To make this proof work, we could change the claim to $n^2+1$ is always even.
				\begin{claim}
					$n^2+n$ is odd.
				\end{claim}
				\begin{proof}
					$n=1$ is odd.
					
					Inductively, assume $n^2+n$ is odd. Then
					$$(n+1)^2+(n+1)=n^2+2n+1+n+1=n^2+n+2(n+1)$$
					so $(n+1)^2+(n+1)$ is the sum of an odd number and an even number, hence itself odd. This completes the inductive step.
				\end{proof}
				In this proof, the writer didn't include even numbers, instead using the formula for odd numbers and finding all numbers in there.
				\begin{claim}
					Every natural number is odd.
				\end{claim}
				\begin{proof}
					$k=1$ is clearly odd.
					
					Inductively, assume $k$ is odd. This means there is an integer $p$ so that $k=2p+1$. Consider $p+1$. Clearly $2(p+1)+1$ is odd.
				\end{proof}
				Didn't use a base case, instead only solving for $n+1$. 
				\begin{claim}
					Every natural number is both even and odd.
				\end{claim}
				\begin{proof}
					Assume that $k$ is both even and odd. Consider $k+1$.
					
					Since $k$ is even, there is $p$ with $k=2p$. So $k+1=2p+1$ is odd. Since $k$ is odd, there is $q$ with $k=2q+1$. So $$k+1=(2q+1)+1=2(q+1)$$is even.
				\end{proof}
                                You didn't do the inductive step, you simply stated that there was a next term, not necessarily that this was divisible by 6.
				\begin{claim}
					$n^3-n$ is divisible by 6.
				\end{claim}
				\begin{proof}
					For the base case: when $n=1$, $n^3-n=0=6\cdot 0$.
					
					Now proceed inductively. Assume that for all $k$, $k^3-k$ is divisible by 6. Then, since $n+1$ is one possible value of $k$, we have that $(n+1)^3-(n+1)$ is divisible by 6. 
				\end{proof}

		\end{enumerate}
	\item[proofs] Prove the following claims. 
          \setcounter{claim}{0}
		\begin{enumerate}
			\item For any natural number $p$, 8 divides $5^{2p}-1$.
\begin{claim}
For any natural number $p$, 8 divides $5^{2p}-1$
\end{claim}
\begin{PBI}
We must first confirm that the base case, or $5^{2p}-1$ is divisible by 8. We determine that the value of the function $5^{2p}-1$ at 1 is 24, which is $8*3$, so this statement is valid. Next, we can see that the solution to this is $8m$, where m is a natural number. We can now use mathematical induction by setting p=n and plugging in p+1 to show that: \\
\begin{align*}
8m &= 5^{2(p+1)}-1 \tag{Solving for p+1, since we know that this is true when n is true when n=1} \\
&= 5^2*5^{2p}-1 \\
&=25(8m+1)-1 \\ \tag{Substituted $5^{2p}$ through inductive assumption}
&=8(25m+3) \\
\end{align*}
Therefore, we can see that by mathematical induction, $5^{2p}-1$ will always be divisible by 8 as long as p is in the natural numbers. \\
\end{PBI}

			\item For any natural number $\ell$, $3^\ell\geq 1+2^\ell$.
			\item Let $a_1,\ldots, a_n$ be real numbers. Then
				$$2^{\left(\sum_{k=1}^n a_k\right)}=\prod_{k=1}^n 2^{a_k}$$ \\
\begin{claim}
Given $a_1,\ldots, a_n$ are real numbers then $2^{\left(\sum_{k=1}^n a_k\right)}=\prod_{k=1}^n 2^{a_k}$ \\
\end{claim}
\begin{PBI}
We can first test our base case by plugging in n=1, which gives us $2^{a_1}=2^{a_1}$, so the base case is correct. We can now set k=r+1, to give us: \\
\begin{align*}
\prod_{r=1}^{n+1} 2^{a_r}&=2^{\left(\sum_{r=1}^n a_{r}\right) +a_{r+1}} \tag{replaces k with r+1} \\
&=2^{a_r+1}*2^{\left(\sum_{r=1}^n a_{r}\right)} \\
&=2^{a_{r+1}}*2^{\left(\sum_{r=1}^n a_{r}\right)} \tag{Substituted using inductive assumption}\\
&=2^{a_{r+1}}*\prod_{r=1}^n 2^{a_r} \\
&=\prod_{r=1}^{n+1} 2^{a_r} \\
\end{align*}
As we can see, these two are equal at n=1 and all subsequent values, so this holds true. This was achieved with mathematical induction. \\
\end{PBI}
			\item ($\star$) For any $k\in\mathbb{N}$, and any real numbers $r,s$, we have 
				\begin{align*}
					r^k-s^k=(r-s)\sum_{\substack{p+q=k-1\\p,q\geq 0}} r^qs^p=(r-s)\left(r^{k-1}+r^{k-2}s+r^{k-3}s^2+\cdots+r^2s^{k-3}+rs^{k-2}+s^{k-1}\right) \\
				\end{align*}
			\item Consider the possible results of flipping a fair coin $n$ times. There are $2^n$ possible outcomes. 
\begin{claim}
Given $n$ coin tosses, there are $2^n$ outcomes.
\end{claim}
\begin{PBI}
Since the amount of flips will be twice the previous amount, as you can do all of the previous flip possibilities on heads and again on tails. We can first determine that $2^n$ does satisfy the base case of one coin flip, which we know has two possibilities. We must then solve for $2^1$, which is 2, therefore verifying the base case. We can now use k with $k=n$ and solve for $k+1$, or $2^{k+1}$ to show that:\\
\begin{align*}
&2^{n+1}\\
&2(2^{n}) \\
\end{align*}
This confirms that $2^n$ accurately describes the probability of heads and tails. Through Mathemartical Induction, we must conclude that given $n$ coin tosses, there are a total of $2^n$ possibilities. $\qed$
\end{PBI}
			\item ($\star$) For any natural number $q$, $$\sum_{i=1}^{2^q}\frac{1}{i}\geq 1+\frac{q}{2}$$
			\item Prove the power rule for derivatives:~for any $n\in\mathbb{N}$, we have $\frac{d}{dx}[x^n]=nx^{n-1}$. You may {\bfseries only} use the product rule for derivatives and the fact that $\frac{d}{dx}x=1$.
\begin{claim}
	The derivative of a function in the form of $x^n=nx^{n-1}$.
\end{claim}
\begin{PBI}
We can confirm the base case by confirming that $\frac{d}{dx}(x^1)=\frac{d}{dx}(x)=1$. Now that we have proven the base case, we can use the inductive step in which $n=k+1$ to show that: \\
\begin{align*}
nx^{n-1}&=\frac{d}{dx}x^{k+1}\\
&= \frac{d}{dx}x(x^k) \\
&= x^k+x\frac{d}{dx}x^k  \tag{Applied product rule} \\
&= x^k+x(kx^{k-1}) \tag{Substituted with the inductive assumption} \\
&= x^k+kx^k \\
&= (k+1)x^{(k+1)-1} \\
\end{align*}
As we can see, the input of k+1 in $\frac{d}{dx}x^{n}$ resulted in the same values being replaced in the second part. Therefore, through proof by mathematical induction, we can confirm that $\frac{d}{dx}(x^n)=nx^{n-1}$. $\qed$
\end{PBI}
			 \item ($\star$) The {\bfseries power rule for integrals}:~for any $n\in\mathbb{N}$, we have 
				\begin{align*}
					\int x^ndx=\frac{1}{n+1}x^{n+1}+C
				\end{align*}
				for some constant $C$.\\
				You {\bfseries may not} use the previous result. You may use {\bfseries only} the following calculus facts: the linearity properties of the integral; $\int C dx=Cx+D$ for some constant $D$; $\frac{d}{dx} x=1$; integration by parts.
			\item $\frac{d^r}{dx^r}x^r=r!$
\begin{claim}
$\frac{d^r}{dx^r}x^r=r!$
\end{claim}
\begin{PBI}s
We must first confirm this is through with the base case n=1. $\frac{d^1}{dx^1}x^1=1!=1$. Now that we know the base case holds true, we can use the inductive step by having $r=k$ and substituting $n$ for $r+1$ to show that:
\begin{align*}
r!&=\frac{d^{k+1}}{dx^{k+1}}x^{k+1} \\
&=\frac{d^k}{dx^k}\frac{d}{dx}x^{k+1} \\
&=\frac{d^k}{dx^k}(k+1)x^k \tag{substituted using inductive assumption} \\
&=(k+1)\frac{d^k}{dx^k}x^k \tag{Used the constant multiple rule of derivatives.} \\
&=(k+1)k! \tag{Substitutes equation given in the claim.} \\
&=(k+1)! \tag{Uses definition of factorial} \\
\end{align*}
\end{PBI}
Therefore, through mathematical induction, $\frac{d^r}{dx^r}x^r=r!$ is true. $\qed$

			\item The {\bfseries constant multiple rule for higher derivatives}: for any function $f$ with at least $n$ derivatives and any constant $c$, we have $\frac{d^n}{dx^n}\left[cf\right]=c\left[\frac{d^n}{dx^n}f\right]$. You may assume the constant multiple rule for derivatives.
\begin{claim}
For any function $f$ with at least $n$ derivatives and any constant $c$, we have $\frac\
{d^n}{dx^n}\left[cf\right]=c\left[\frac{d^n}{dx^n}f\right]$.
\end{claim}
\begin{PBI}
We can first prove the base case by seeing that $\frac{d^1}{dx^1}[cf]=c\frac{d}{dx}f$, where f is any function, due to the constant multiple rule for derivatives. Next, we can use $k=n$ and substitute $k+1$  to perform the inductive step: \\
\begin{align*}
c \frac{d^{n}}{dx^{n}}[f]&=\frac{d^{k+1}}{dx^{k+1}}[cf] \\
&=\frac{d}{dx}(\frac{d^{k}}{dx^{k}}[cf]) \\
&=\frac{d}{dx}(c\frac{d^k}{dx^k}[f]) \tag{Substituted using equation given in claim} \\
&=c\frac{d}{dx}(\frac{d^k}{dx^k}[f]) \tag{Used constant multiple rule for derivatives} \\
&=c\frac{d^{k+1}}{d^{k+1}}[f] \\
\end{align*}
Therefore, we can see this works for n as well as n+1, therefore through mathematical induction $\frac{d^n}{dx^n}\left[cf\right]=c\left[\frac{d^n}{dx^n}f\right]$. $\qed$
\end{PBI}
			\item The {\bfseries sum rule for higher derivatives}: for any functions $f$, $g$ with at least $n$ derivatives, we have $\frac{d^n}{dx^n}\left[f+g\right]=\frac{d^n}{dx^n}f+\frac{d^n}{dx^n}g$. You may assume the sum rule for derivatives.
                          \begin{claim}
                            for any functions $f$, $g$ with at least $n$ derivatives, we have $\frac{d^n}{dx^n}\left[f+g\right]=\frac{d^n}{dx^n}f+\frac{d^n}{dx^n}g$.
                            \end{claim}
                          \begin{PBI}
                            We must first prove the base case, or with n=1. We can see that  $\frac{d}{dx}\left[f+g\right]=\frac{d}{dx}f+\frac{d}{dx}g$ is true due to the sum rule for derivatives. Now that we know this is true, we can use k=n and solve for k+1 to show that: \\
                            \begin{align*}
                              \frac{d^{k+1}}{dx^{k+1}}f+\frac{d^{k+1}}{dx^{k+1}}g &= \frac{d^{k+1}}{dx^{k+1}}[f+g] \\
                              &=\frac{d}{dx}\frac{d^k}{dx^k}[f+g] \\
                              &=\frac{d}{dx}\frac{d^k}{dx^k}f+\frac{d}{dx^k}g \tag{Substituted using the inductive assumption}\\
                              &=\frac{d^{k+1}}{dx^{k+1}}f+\frac{d^{k+1}}{dx^{k+1}}g \\
                              \end{align*}
                            Therefore, since $\frac{d^n}{dx^n}\left[f+g\right]=\frac{d^n}{dx^n}f+\frac{d^n}{dx^n}g$ is true in the base case of n=1 and for n+1, through mathematical induction this is true. $\qed$


                          \end{PBI}
			\item ($\star$) The {\bfseries Binomial Theorem}: for any real numbers $x,y$, and any $n\in\mathbb{N}$,
				\begin{equation*}
					(x+y)^n=B_{n,0}x^n +B_{n,1}x^{n-1}y+B_{n,2}x^{n-2}y^2+\cdots+B_{n,n-2}x^2y^{n-2}+B_{n,n-1}x y^{n-1}+B_{n,n}y^n
				\end{equation*}
				({\em Hint.} At some point you will need to ``combine like terms".)
			\item ($\star$) Any legal word has more $a$s than $b$s.
\begin{claim}
Any legal word has more $a$s than $b$s.
\end{claim}
\begin{PBI}
We can see through the base case, aba, that there are more $a$s than $b$s. Next, we must realize that all of the transformations (legal alterations of an existing legal word which provide another legal word) are the same in terms of the amount of $a$s and $b$s added, so if can be proven for any of the transformations that there are more $a$s and $b$s, then it is true for all of them. These transformations can be represented as an inductive step by using a recursive model and determining what happens between step n and step n+1(n represents a selected step in the natural numbers). Since all of them offer an equal amount of $a$s and $b$s, and the base case has an equal amount of them, there will always be more $a$s than $b$s by mathematical induction. $\qed$ 
\end{PBI}
			\end{enumerate}
\end{description}
\end{document}
