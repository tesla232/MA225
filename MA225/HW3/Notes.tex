\documentclass{article}
\usepackage{amsmath}
\begin{document}
\newtheorem{PBI}{Proof By Induction}
\newtheorem{claim*}{claim}
\begin{center}
Notes for 1/31/17
\end{center}

An inductive prood od a statement of a statement of the form $\forall \in N, P(n)$ goes as follows:
\begin{enumerate}
\item Prove P(1). (This is usually pretty obvious)
\item prove that $\forall n,(P(n) \Rightarrow P(n+1)$
\begin{enumerate}
\item Let n be any natural number
\item Assume P(n) is true
\item ...Juicy bits...
\item Conclude P(n+1) is true
\end{enumerate}
\end{enumerate}
\begin{claim*}
The square of any odd number has a for 8k+1 for some interger k.\\
Claim: (Case when the off number is at least 3)\\
\end{claim*}
\begin{PBI}
$\forall n \in N \exists k:(2n+1)^2=8k+1$ \\
We proceed by induction.\\
Base Case (n=1) Use k=1. Then $(2*1+1)^2=9=8k+1$ \\
Induction step: Let n be a natural number. Assume $\exists k:(2n+1)^2=8k+1$.\\
(Scratch work)\\
$P(n): \exists k: (2n+1)^2=8k+1$\\
$P(1):$ \\
$P(n) \Rightarrow P(n+1$) \\
$P(n+1): \exists l: (2(n+1)+1)^2=8l+1$ \\
(End Scratchwork)\\
\begin{align*}
(2(n+1)+1)^2 &= (2n+2+1)^2 \\
&=((2n+1)^+2)^2 \\
&=(2n+1)^2)+4+4(2n+1)\\
&=8k+1+4(2n+1+1)\\
&=8k+1+4(2n+2) \\
&=8k+1+8(n+1) \\
&=8)k+n+1)+1 \\
&=8l+1 \\
\end{align*}
Therefore, through mathematical induction, the claim is true. \\
\end{PBI}
\begin{claim*}
For any $n \in N, \frac{n^3}{3}+\frac{n^5}{5}+\frac{7n}{15}$ is an integer.\\

\end{claim*}

\begin{PBI}
NTS:$\frac{1}{3} +\frac{1}{5} + \frac{1}{5} + \frac{7}{15} =1$ \\
Inductive Step: Let $n\in N$ Assume $\frac{n^3}{3}+\frac{n^5}{5}+\frac{7n}{15}$ is an Integer. \\
WE'll show  $\frac{(n+1)^3}{3}+\frac{(n+1)^5}{5}+\frac{7(n+1)}{15}$ is an integer. \\
\begin{align*}
\frac{(n+1)^3}{3}+\frac{(n+1)^5}{5}+\frac{7(n+1)}{15} &= \frac{3}{n^3+3n^2+3n+1}{3}+\frac{n65+5n^4+10n^3+10n^2+5n+1}{5}}+\frac{7n+1}{15} \\
&=\frac{n^3}{3}+\frac{n^5}{5}+\frac{7n}{15}+n^2+n+n^4+2n^3+2n^2+n+\frac{1}{3}+\frac{1}{5}+\frac{7}{15}\\
\end{align*}
This is an Integer. By induction, this completes the proof.
\end{PBI}
\begin{claim*}
Any finite collection of real numbers has a greatest element.\\ 
For any $n \in N$, any set of exactlty n real numbers has a greatest element.
\end{claim*}
\begin{PBI}
( By induction on the size of set n)\\
Base case n=1. Letn be a set of exactly 1 real number. \\


Let a be Franklin single element, then a is the greatest.\\
Inductive step. Assume that any set of exactly n distinct real numbers has a greatest element.\\
We'll show: any collection of exactly n+!distinct real numbers has a greatest element \\
Let B be a set of exactly n+1 distinct real numbers. \\
Let b be one of them. Let B' be B, but with b removed. Then B' has exactly n distinct elements.  \\
So B' has a greatest element, Aisha.\\
Case: b<Aisha     Aisha is greatest in B \\
Case: Aisha<b. b is greatest in B. \\

\end{PBI}
\end{document}
