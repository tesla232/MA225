\documentclass{letter}
\usepackage{amsmath,amsthm}
\usepackage{amsfonts}
\usepackage{amssymb}
\usepackage{enumerate}
\usepackage[margin=.9in]{geometry}
\usepackage{cite}
\usepackage{asymptote}



\newtheorem*{theorem*}{Forbidden Fruit Theorem}
\newtheorem*{allowed}{Allowed Fact}

\theoremstyle{definition}
\newtheorem*{definition}{Definition}

\begin{document}
\pagestyle{empty}

{\Large MA 225 Problem Set 2: logic 2}\\
Jordan Paldino \\
(Note: throughout the document, distribution through a commaed list is assumed. for instance, $\forall a,b$ means $\forall a, \forall b$)
\begin{description}
\item[facts and definitions] The following fact is {\em true}, but we have not proved it so you {\bfseries may not use it} anywhere in this homework set:
\begin{theorem*}
	Every integer is either even or odd.
\end{theorem*}

You {\bfseries may} use the following fact:
\begin{allowed}
	1 is not an even number.
\end{allowed}

You will also need this definition:
\begin{definition}
	We call the function $f$ an {\em odd function} if $$\forall x, f(-x)=-f(x)$$  
	
	We call the function $f$ an {\em even function} if $$\forall x, f(-x)=f(x)$$
\end{definition}
\item[exercises] These problems don't require you to write proofs.
\begin{enumerate}
	\item For proofs 1-5, write a symbolic version of each claim. Your symbolic version may only use logical symbols, variables, numbers, and the symbols $+,-,=$.
	\item Let $H(t)$ represent {\em $t$ is happy}. Let $S(x)$ represent {\em $x$ is short}.
		\begin{enumerate}
			\item Compare the meanings of $\forall t, \left[H(t)\vee \sim H(t)\right]$ and $\left[\forall t, H(t)\right] \vee \left[\forall t,\sim H(t)\right]$.
The first statement proposes that for all t, H(t) is happy or not happy, which is a tautology. The other states that for all t in H(t), all cases are happy, while for all cases of t this is not the case. This essentially means that none or all of t return happy for H(t) 
			\item Compare the meanings of $\forall t, \left[H(t)\wedge S(t)\right]$ and $\left[\forall t, H(t)\right]\wedge \left[\forall t, S(t)\right]$.
They mean the same thing,for all t $H(t)$ and $S(t)$ are true.
			\item What's the general lesson about how $\vee$ and $\wedge$ interact with $\forall$? \\
                          That $\forall$ distributes over $\wedge$ but not $\vee$
		\end{enumerate} 
	\item Let $H(t)$ represent {\em $t$ is happy}. Let $S(x)$ represent {\em $x$ is short}.
		\begin{enumerate}
			\item Compare the meanings of $\exists x:\left[H(x)\vee \sim H(x)\right]$ and $\left[\exists x: H(x)\right] \vee \left[\exists x:\ \sim H(x)\right]$.
                          They both mean the same thing that is that there exists a value of t which is true or false.
			\item Compare the meanings of $\exists x: \left[H(x)\wedge S(x)\right]$ and $\left[\exists x: H(x)\right]\wedge \left[\exists x: S(x)\right]$. The first one means that there exists a value of t that $H(x)$ and $S(x)$ are true. The second one means that there is a value of x that returns true in both H and S, but it isn't necessarily the same x.
                          
			\item What's the general lesson about how $\vee$ and $\wedge$ interact with $\exists$? It doesn't distribute over $\wedge$ but it does over $\vee$
		\end{enumerate}
		\item Consider the following line from William Shakespeare's {\em The Merchant of Venice}:
			\begin{center}{All that glisters is not gold.}\end{center}
			\begin{enumerate}
				\item Use $G(t)$ to represent {\em $t$ glisters} and $g(x)$ to represent {\em $x$ is gold}. Write a symbolic sentence to represent {\bfseries the literal meaning} of Shakespeare's line. \\$\forall t:(\sim G(t))$
				\item Write a symbolic sentence which represents {\bfseries the meaning Shakespeare wants us to take} from the line. ({\em Hint.} If you find this request confusing, try revisiting your previous answer.)\\ $ \exists t:(\sim G(t))$
				\item Apply logical rules to simplify your symbolic answer to part (b). Then translate this simplified version into English. There is something that glisters but is not gold.\\
			\end{enumerate}
		\item The following joke by writer Joyce Carol Oates relies on an equivocation between two plausible interpretations of the original States United tweet. Distinguish between the two interpretations by expressing each as a symbolic sentence. Use $L(x,y,z)$ to represent {\em $x$ shot $y$ in $z$}, where the universe for $x$ is all toddlers, the universe for $y$ is all people, and the universe for $z$ is all weeks. \\
Oate's Interpretation: $L(t): \exists x: ( x \wedge \wedge y \wedge) \wedge \forall z$ \\
States united: $L(t): \exists x \wedge \exists y \exists \wedge \forall z $
		\item Consider the two claims
			\begin{enumerate}[(1)]
				\item {\em Every time I've been to Vegas, I was high on cocaine the whole time.}
				\item {\em I have never done illegal narcotics.}
			\end{enumerate}
			Explain how it is that these claims could both be true (when spoken by the same person). What is the general lesson about the relationship between the claims $\forall x, P(x)$ and $\exists x: P(x)$?You have never been to Vegas. It means that even if a statement is true for all parts of t, it doesn't necesarily imply existance.
\end{enumerate}
\item[proofs] Write a complete proof of each of the following statements.
	\begin{enumerate}
		\item The product of any integer with any even integer is even.
$\forall a, b,\exists k, m:(b=2k \Rightarrow ab=2m$ \\
\underline{Claim:} Given any integer a, the product, p with an even integer b is even. \\
\textbf{Proof:} Given b is an even integer, we can determine that it can be writen as $b=2k$, where k is an integer that makes $b=2k$ true.We also define r to be $r=ka$.
\begin{align*}
p &= ba \\
 &= 2ka \tag{Replaces b with 2k} \\
 &= 2r \tag{Uses r to represent ka} \\
\end{align*} 
Therefore, as we can see, p=2r, and since this fits the formula 2j where j is an integer, ab where b is even, the product will be even.
		\item The difference of two odd integers is always even.
$  \forall j, k \exists e:((2j+1)+2k+1 \Rightarrow 2e)) $ \\
Claim: Given two odd numbers j and k, the difference D will always be even. \\
Proof: We can represent j as $j=2l+1$, where l is an integer which makes the statement true, and $k=2o+1$ where $o$ is an integer which which makes the statement true. Furthermore, we define $q$ to be $q=o-p$ We can use this to see that:
\begin{align*}
D &= j-k \\
&= (2o+1)-(2p+1) \tag{Replaced j with 2o+1 and k with 2p+1} \\
&= 2o-2p+1-1 \tag{distributed negative one} \\
&= 2(o-p) \tag{Subtracted ones and factored out 2.} \\
D &= 2q \tag{rewrote (o-p) as q} \\
\end{align*}
Therefore, D is in the form of 2l=n, which means that D is even. \\
		\item If the sum, S of two integers is odd, their difference, D is odd. \\
$\forall a,b, \exists j,k:(a+b=2j+1 \Rightarrow a+b=2j-1)$ \\
Claim: If the sum of two integers, a and b is odd, then their difference is odd. \\
Proof: We must first determine all of the cases which the sum of two numbers is odd.\\
Case 1 even odd: \\
\begin{align*}
S &= 2k+1+2j \\
&= 2(k+j)+1 \tag{Factored out two} \\
&= 2r+1 \tag{Set r=k+j} \\
\end{align*}
Since this is odd, we can see if the difference is odd. \\
\begin{align*}
D &= 2k+1-(2j) \\
&= 2(k-j)+1 \tag{Factor out 2} \\
&= 2m+1 \tag{Set k-j equal to m} \\
\end{align*} 
We can see now that in this case, the claim remians true. We know that this remains true with subtracting an even and odd because we can simply multiply -1 by both sides and change the value of D without affecting if it is even. \\
Case 2 odd odd: \\
We can see that this produces an even number when added together, as seen below: \\
\begin{align*}
S &= 2k+1+2j+1 \\
&=2(k+j+1) \tag{Factored out the 2}\\
&= 2p \tag{set p as k+j+1} \\
\end{align*}
As we can see, this results in an even answer, therefore this case isn't considered for the claim. \\
Case 3 Even Even: \\
The sum of two even numbers will produce an even number, as seen below:
\begin{align*}
S &= 2j+2k \\
S &= 2(j+k) \tag{Factored out 2} \\
S &= 2f \tag{Set f equal to j+k} \\
\end{align*}
As we can see, the only case which the sum of two integers is even is if there is an even number and an odd function, and the difference of these numbers also produces and odd number, therefore if the sum of two integers, a and b is odd, then their difference is odd.






		\item ($\star \star$) No integer is both even and odd. \\
$\forall a, b: \sim (a=2k \wedge a=2j+1)$ \\
Claim: No Integer is both even and odd \\
Proof: Let us assume that there exists an integer, $I$, which are equal to both 2k and 2j+1 where there exists a j and k that are integers indicating that I is both even are odd (We use different variables to consider different values being used).We also define m as m=k-j We must set these values equal to eachother in order to see:\\
\begin{align*}
2k &= 2j+1 \\
-1 &= 2j-2k \tag{subtracts 2k and -1 from both sides}\\
1 &= 2(k-j)\tag{multiplies -1 by both sides and factors out 2 from 2j-2k}\\
1 &= 2m \tag{Represents k-j as m}
\end{align*}
Since we know that 1 is not an even number, and the above insinuates this, we know that $I$ does not exist. \\

		\item If $m$ divides $a$ and $n$ divides $b$, then $mn$ divides $ab$. \\
$ \exists a,m,k,j,b: (a=m(k) \wedge b=n(j) \Leftrightarrow \exists r: ab=mn(r))$ \\
Claim: If $m$ is a factor of $a$ and $n$ is a factor of $b$, then $mn$ is a factor of $ab$.
Proof: We can first define $a$ as $a=mj$ and $b$ as $b=nk$ with j and k being integers. Next, we define o to be o=jk. Once this is done, we can rewrite ab as: \\
\begin{align*}
ab&=nkmj \\
&=mn(jk) \\
ab &= mn(o) \tag{substitute jk for o} \\
\end{align*}
As we can see, mn is a factor of ab, as it is the product of mn and o. \\
                \item ($\star$)  Given an integer $t$, if there are integers $m$ and $n$ so that $15m+16n=t$, then there are integers $r$ and $s$ so that $3r+8s=t$. \\
Claim:  Given an integer $t$, if there are integers $m$ and $n$ so that $15m+16n=t$, then there are integers $r$ and $s$ so that $3r+8s=t$. \\
Proof: Since m and n must be integers, as well as r and s, we can set r=5m and s=2n, which satisfies the conditions of integers. \\
\begin{align*}
t &= 3r + 8s \\
&= 3(5m)+8(2n) \tag{plugged in values of r and s} \\
t &= 15m +16n \tag{Solved for the values above} \\
\end{align*}
Since we know this to be true, there are integers which make 3r+8s=t when 15m+16n=t. \\
		\item ($\star$) If there are integers $m$ and $n$ with $12m+15n=1$, then $m$ and $n$ are both positive. \\
Claim: If there are integers $m$ and $n$ with $12m+15n=1$, then $m$ and $n$ are both positive. \\
We must first show that $12m+15n=1$ is not valid for any integer. We can first factor out 3 from the equation, giving us: \\
\begin{align*}
 1 &= 3(4m+5n) \\
\end{align*}
Since 3 is not a factor of 1, there is no integer value which works, as 1 does not have 3 as a factor. Therefore the statement is valid, as it doesn't satisfy the initial condition. So, if there are integers $m$ and $n$ with $12m+15n=1$, then $m$ and $n$ are both positive, as there are no integers which make that statement true.  \\
		\item ($\star$) The sum of two odd functions is [BLANK]. (Fill in the blank, and prove your answer.) \\
Claim: The sum of two odd functions is odd. \\
Proof: Let's consider two odd functions, $f(x)$ and $g(x)$ which $x$ is an input to $f$ and for all $x$, $f(-x)=-f(x)$ and $g(-x)=-g(x)$: \\
\begin{align*}
f(-x) + g(-x) &= -f(x) + -g(x) \tag{addition of two functions} \\
(f+g)(-x) &= -(f+g)(x) \tag{factor our (-x) from left side and (-x) and -1 from right side} \\
h(-x) &= -h(x) \tag{Set f+g=h} \\
\end{align*}
Therefore, the sum of two odd functions is odd. \\

		\item ($\star \star$) Show that there is a unique function which is both even and odd. \\
Claim: There is a unique function that is both even and odd. \\
Proof: Let us consider the class of even functions $f(-x)=f(x)$ and odd functions $f(-x)=-f(x)$. Let us also define a function as being an equation in which each input produces a unique output. Let us now define $-f(x)$ graphically as the reflection of inputs across the x axis, as the distance between zero stays equidistant, however the direction in which these are is opposite. So, for a function to be both even and odd means for the outputs of $f(-x)$ must equal both $f(-x)$ and $-f(-x)$. Because functions have unique outputs, $-f(-x)$ must equal $f(-x)$, or an equation in which the reflection and original converge. Because the only value which this holds true is 0, as this serves as the axis of reflection, and when you use $f(x)=0$ (implies $f(-x)$=0, as it is not dependent on $x$), this results in $f(-x)=-f(x)$, $0=-0$, which is true, and $f(-x)=f(x)$, $0=0$, which is also true. Therefore, the only function which would satisfy the conditions is $f(x)=0$.  
\end{enumerate}
\end{description}
\end{document}

