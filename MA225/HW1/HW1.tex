
\documentclass{letter}
\usepackage{amsmath,amsthm}
\usepackage{amsfonts}
\usepackage{amssymb}
\usepackage{enumerate}
\usepackage[margin=.9in]{geometry}
\usepackage{cite}
\usepackage{asymptote}

\newtheorem*{theorem*}{Theorem}

\begin{document}
\pagestyle{empty}

{\Large MA 225 Problem Set 1: logic 1}\\
\begin{description}
\item[exercises] These problems don't require you to write proofs.
\begin{enumerate}
	\item We will show that although it's nice to have lots of connectives, we don't actually {\em need} them all.
			\begin{enumerate}\itemsep=1.25mm
				\item Express the following formul\ae\  using only the symbols $P$, $Q$, $\sim$, and $\wedge$: \\
				$P\vee Q$, $P\Rightarrow Q$, $P\Leftrightarrow Q$ \\
                               $P \vee Q: \sim (\sim P \wedge \sim Q)$ \\
                                  $P \Rightarrow Q: \sim (P \wedge \sim Q))$ \\
                                  $P \Leftrightarrow Q: \sim(\sim P \wedge Q) \wedge \sim (P \wedge \sim Q)$
				\item Express the following formul\ae\  using only the symbols $P$, $Q$, $\sim$, and $\vee$:\\
					$P \wedge Q$, $P\Rightarrow Q$, $P\Leftrightarrow Q$ \\
                                  $P \wedge Q: \sim ( \sim (P \vee Q) \vee \sim ( P \vee \sim Q) \vee \sim ( \sim P \vee Q))$ \\
                                  $P \Rightarrow Q: \sim P \vee Q $ \\
                                  $P \Leftrightarrow Q: \sim ((\sim P \vee Q) \vee \sim (P \vee \sim Q))$ \\
				\item Express the following formul\ae\ using only the symbols $P$, $Q$, $\sim$, and $\Rightarrow$\\
				$P\wedge Q$, $P\vee Q$, $P\Leftrightarrow Q$ \\
                                  $P\wedge Q: \sim (P \Rightarrow \sim Q)$ \\
                                  $P \vee Q: (\sim P \Rightarrow  Q)$ \\
                                  $P \Leftrightarrow Q: (P \Rightarrow \sim Q) \Rightarrow \sim (\sim P \Rightarrow  (\sim Q \Rightarrow P)) $ \\
                                
				\item Explain why this means we only need $\sim$ and {\em one} of $\wedge$, $\vee$, and $\Rightarrow$. \\
                                   Because we have shown above that we can represent each symbol with a single symbol
			\end{enumerate}
	\item Define the connective $\veebar$ so that $P\veebar Q$ is true exactly when exactly one of $P$ and $Q$ is true.
			\begin{enumerate}\itemsep=1.25mm
				\item Make a truth table for $P\veebar Q$.
                                  \begin{center}
                                    \begin{tabular}{||c|c|c||}
                                    \hline
                                    P & Q & P $\veebar$ Q \\ \hline
                                    True & True & False \\ \hline
                                    True & False & True \\ \hline
                                    False & True & True \\ \hline
                                    False & False & False \\ 
                                    \hline
                                    \end{tabular}
                                    \end{center}


                                    
                                    
                                    
				\item Show that $P\veebar Q$ is equivalent to $(P\vee Q)\wedge (\sim (P\wedge Q))$.$\veebar$ by definition means that while the logical statement means the statement has a single true statement, such as is possible with an or statement, it lacks the double truth as in a and statement. By making the statement as it is, it requires one of the arguments to be true but doesn't allow both.
				\item Express $\sim (P\veebar Q)$ in terms of $\sim$, $\vee$, and $\wedge$. \\
                                  $\sim(\sim P \wedge Q) \wedge \sim (P \wedge \sim Q)$
 
			\end{enumerate}
	\item Make a truth table for $P\veebar Q\veebar R$.
                                  \begin{center}
                                    \begin{tabular}{||c|c|c|c|c||}
                                    \hline
                                    P & Q & R & P $\veebar$ Q & (P $\veebar$ Q) $\veebar$ R\\ \hline
                                    True & True & True & False & True\\ \hline
                                    True & True & False & False & False\\ \hline
                                    True & False & True & True & True\\ \hline
                                    True & False & Flase & True & True\\ \hline
                                    False & True & True & True & False\\ \hline
                                    False & True & False & True & True\\ \hline
                                    False & False & True  & False & True\\ \hline 
                                    False & False & False  & False& False\\ 
                                    \hline
                                    \end{tabular}
                                    \end{center}
	\item For each of the following, identify the antecedent and the consequent. Then indicate whether the statement is true or false. \\
		
                  
                         ++Antecedent++  --consequent-- \\
                         \begin{enumerate}
			\item  --The Nile River flows east-- only if ++64 is a perfect square++. \\ 
                         This statement is false, as the statement suggests the Nile flows east if 64 is a perfect square, which can be seen with $8^2$. \\ 
			\item  ++$1+1=2$++ is sufficient for -- 3>6 --. \\ 
                         This statement is false, as while 1+1=2, it states that this true fact is enough to determine $3>6$, which is false. \\
			\item  If ++Euclid's birthday was April 2++, then --rectangles have four sides.-- \\
                         This statement is true, as it does not say that this is the only time when rectangles have four sides, and since they have four sides, this statement is true. \\
			\item  ++If squares have three sides++, then --triangles have four sides--. \\
                          This statement is true, as while triangles don't have four sides, this only applies when squares have three sides, which is false. \\
			\item Fish bite-- only when ++the moon is full++. \\
                          This is false, as plenty of people catch fish during the day when the moon isn't even out. \\
			\item An indictment++ is necessary for --a conviction--. \\
                          This is true, as without an indictment, there would be nothing to convict the person of. \\
	                  
	\end{enumerate}
	\item Consider each of the following sentences as you would understand them if you heard it on the street. Identify, for each sentence, the antecedent and the consequent.
			\begin{enumerate}\itemsep=1.25mm
				\item --I will go to the store-- ++unless it is raining++. On the condition that it is not raining, I will go to the store. \\
				\item --The Dolphins will not make the playoffs-- ++unless the Bears win all the rest of their games++. The only way the Dolphins will make the play Offs is given they win the rest of their games.\\
				\item  --You cannot go to the game-- ++unless you do your homework first++. Doing your homework is required to go to the game. \\
				\item  --You won't win the lottery-- ++unless you buy a ticket++. You need to buy a ticket in order to win the lottery.\\
			\end{enumerate}
	\item In each of the previous problem's sentences, use a different conditional keyword to express the sentence. You may {\bfseries not} use {\em if. . . then}. {\bfseries Be sure your rephrasing agrees with your answer in the previous problem!} \\
	\item Which of the following are tautologies? Which are contradictions? For each, give an explanation that uses a truth table {\bfseries and} an explanation that does not use a truth table. ({\em Hint.} Try expressing in words what each says.)
			\begin{enumerate}\itemsep=1.25mm
				\item $(\alpha\wedge \gamma)\vee \left[(\sim \alpha)\wedge(\sim\gamma)\right]$
                                  \begin{center}
                                  \begin{tabular}{||c|c|c||}
                                    \hline
                                    $\alpha$ & $\gamma$ & $(\alpha\wedge \gamma)\vee \left[(\sim \alpha)\wedge(\sim\gamma)\right]$ \\ \hline
                                    True & True & True \\ \hline
                                    True & False & False \\ \hline
                                    False & True & False \\ \hline
                                    False & False & True \\ \hline
                                    \end{tabular}
                                    \end{center}

				\item $\sim\left[P\wedge (\sim P)\right]$ \\
                                  \begin{center}
                                  \begin{tabular}{||c|c||}
                                    \hline
                                    $P$ & $\sim\left[P\wedge (\sim P)\right]$ \\ \hline
                                    True & True \\ \hline
                                    True & True \\ \hline
             
                                    \end{tabular}
                                    \end{center}
				\item $(\Psi \wedge \Phi)\vee \left[(\sim \Psi)\vee(\sim \Phi)\right]$ \\ 
                                \begin{center}
                                  \begin{tabular}{||c|c|c||}
                                    \hline
                                    $\Psi$ & $\Phi$ & $(\Psi \wedge \Phi)\vee \left[(\sim \Psi)\vee(\sim \Phi)\right]$ \\ \hline
                                    True & True & True \\ \hline
                                    True & False & True \\ \hline
                                    False & True & True \\ \hline
                                    False & False & True \\ \hline
                                    \end{tabular}
                                    \end{center}
				\item $[A\wedge B]\vee [A\wedge (\sim B)] \vee [(\sim A)\wedge B] \vee [(\sim A)\wedge (\sim B)]$ \\ 
                                  \begin{center}
                                  \begin{tabular}{||c|c|c||}
                                    \hline
                                     A & B &  $[A\wedge B]\vee [A\wedge (\sim B)] \vee [(\sim A)\wedge B] \vee [(\sim A)\wedge (\sim B)]$ \\ \hline
                                    True & True & True \\ \hline
                                    True & False & True \\ \hline
                                    False & True & True \\ \hline
                                    False & False & True \\ \hline
                                    \end{tabular}
                                    \end{center}
			\end{enumerate}
	\item Submit part 4 of the worksheet {\em Useful Logical Facts}.
\end{enumerate}
\item[proofs] Write a complete proof for each of the following statements.
	\begin{enumerate}
		\item $\veebar$ is associative. \\
                  Claim: $\veebar$ is an associative operator \\
                  Proof: Let's consider the following truth table \\
                  \begin{center}
                                    \begin{tabular}{||c|c|c|c|c||}
                                      1 & 2 & 3 & 4 & 5 \\
                                    \hline
                                    P & Q & R & P $\veebar$ (Q $\veebar$ R) & (P $\veebar$ Q) $\veebar$ R\\ \hline
                                    True & True & True & True & True\\ \hline
                                    True & True & False & False & False\\ \hline
                                    True & False & True & True & True\\ \hline
                                    True & False & Flase & True & True\\ \hline
                                    False & True & True & False & False\\ \hline
                                    False & True & False & True & True\\ \hline
                                    False & False & True  & True & True\\ \hline
                                    False & False & False  & False& False\\
                                    \hline
                                    \end{tabular}
                                    \end{center} 
                  As we can see from Columns 4 and 5, the shift in brackets created an identical equation, therefore the equations $P \veebar (Q \veebar R) and (P \veebar Q) \veebar R$ are equal, and the operator $\veebar$ is associative.
		\item All the claims in the worksheet {\em Useful Logical Facts}.
	\end{enumerate}
\end{description}
\end{document}

